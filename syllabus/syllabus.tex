\documentclass{article}
\usepackage[margin=1in]{geometry}
\usepackage{enumitem}
\usepackage{hyperref}
\usepackage{fancyvrb}
\usepackage{multicol}
\usepackage{color}
\usepackage{longtable}

\setlength{\parindent}{0pt}
\setlength{\parskip}{1ex}


\newcommand{\myline}{\end{description}\hrulefill\begin{description}}
\newcommand{\be}{\mbox{}\begin{itemize}}
\newcommand{\ee}{\end{itemize}}

           
\begin{document}
\centerline{\Large\bf CSCI 241: Data Structures}
\centerline{\large\bf Syllabus, Spring, 2017}


\begin{description}


\item[Instructor:] Geoffrey Matthews\\
{\em Email:} {\tt geoffrey dot matthews at wwu dot edu}\\
{\em Office hours:} MTWF 10:00-10:50, CF 469


\item[Lectures:] MTWF 11:00-11:50, CF 226


\item[Catalog copy:] Design and implementation of hash tables, general
  trees, search trees, balanced trees and graphs. Comparison of
  sorting algorithms. Demonstration of the use of data structures in
  various applications. Evaluation of the best data structure for a
  particular task. Programming is required in implementation of
  concepts.

\item[Goals:]  On completion of this course, students will demonstrate:
\begin{itemize}
\item
  Basic understanding of classic data structures including trees, hash
  tables and graphs. 
\item Basic understanding of various sorting algorithms.
\item Thorough understanding of recursion in the definition of and
  operations on these data structures. 
\item The ability to select and design the proper data structures to
  problems requiring complex data structures. 
\item The ability to select the proper sorting algorithms for a
  problem. 
\item The ability to make judgments about the selected data structures
  for a problem. 
\item The ability to implement the introduced data structures and
  sorting algorithms. 

\end{itemize}

\item[Websites:]\mbox{}
\begin{itemize}
  \item For class materials:
    \url{https://github.com/geofmatthews/csci241} 
  \item For turning in homework and grading:
    \url{https://wwu.instructure.com/}  
\end{itemize}

\item[Texts:] The following texts will be used as resources, but no
  problems will be assigned from either book.  Online reading will
  also be assigned as topics come up.
\begin{itemize}
\item {\em Building Java Programs: A Back to Basics Approach, 4th
  edition}, Reges and Stepp.

  You may already have this book from CSCI 145.  Access to online
  content associated with the book is not required.  This is an
  excellent introduction to Java, but does not cover all the material
  for CSCI 241.

  \item {\em Introduction to Algorithms, 3rd edition}, Cormen {\em et
    al.}

    You will need this book for CSCI 305 and CSCI 405, and it is an
    excellent resource on algorithms and advanced data structures.
    
\end{itemize}
\item[Software:] The labs already have Oracle's Java Development Kit
  (JDK) 7 as well as jGrasp (and other IDEs) installed for both
  Windows and Linux.  If you would like to work on your own computer,
  you are welcome to download and install jGrasp
  \url{http://www.jgrasp.org/}.

\item[Programming assignments:] Programs must be turned in on canvas.
  Specifications must be followed exactly (do not rename classes or
  procedures because you think your names are better, do not change
  the signature of the methods, {\em etc.}).
  No late work will be accepted.  It is the student's responsibility
  to make sure the {\em correct} file is submitted by midnight of
  the due date.


\item[Quizzes:]  There will be a short quiz every Friday except the
  first and last weeks of class.  That means every Friday in April and
  May.  Quizzes will be closed book.

\item[Final exam:] Monday, June 5, 8:00-10:00am.
  
The final exam is closed book, with the exception that you may consult
two pieces of paper during the exam.  You may write or print whatever
you wish on each side of these pieces of paper.

\item[Emergencies:] If you have to miss an assignment deadline or a
  quiz because of a medical or other emergency notify me as soon as
  possible and we will negotiate a substitute.  If you know in advance
  that you have to miss a deadline, let me know at least a week in
  advance.

\item[Grading:] Students will be graded on the three programming
  assignments, the eight quizzes, some in-class pop quizzes (which are
  unscheduled and which we will solve together in class), and the
  final exam.

Grades will be assigned based on scores as shown.  At the discretion
of the instructor, scores may be scaled.  Awarding $\pm$ is at the
discretion of the instructor.

\begin{tabular}{|c|c|c|c|c|c|}\hline
Program 1 & Program 2 & Program 3 & Quizzes & Pop quizzes & Final \\\hline
10\% & 20\% & 20\% & 20\% & 5\% & 25\% \\\hline
\end{tabular}

\begin{tabular}{|c|c|c|c|c|c|}\hline
\% & 90-100 & 80-89 & 70-79 & 60-69 & 0-60\\\hline
Grade & A & B & C & D & F\\\hline
\end{tabular}


\item[Academic dishonesty:] Please read Appendix D of WWU's Catalog on
  Academic Dishonesty.  It is available online at
  \url{http://catalog.wwu.edu}.

  Unless specified otherwise, all work for this course is meant to
  be done {\bf individually.}  The work that you turn in for a grade
  must be completely your own, or you will be guilty of academic
  dishonesty and could receive an F for the course.

  However, it can be a valiable learning experience to discuss
  work with your fellow students, and this is encouraged.
  However, after working with a colleague, {\bf you may not keep any
    paper or electronic copies of anything you produced together!}
  You may only keep your memories.  In particular, this means that
  {\bf you may not ask for or give help while sitting in front of a
    computer where the assignment is open!}  Also, {\bf you may not
    use anything a colleague has emailed to you!}  Delete the email
  and do not save a copy.

  To help understand what I mean, remember the {\fbox{\bf Long Term
    Memory Rule}}.  You may discuss, sketch, write things down, use
  your computers, whatever, but after you are done working with your
  fellow students all files must be deleted, whiteboards erased, and
  all papers you created must be destroyed.  You should then watch a
  rerun of {\em the Simpson's}, play a game of ping-pong, take a walk,
  or something else for half an hour. After this you can go back to
  your assignment (alone) and use the knowledge you have now gained.

  It is very easy for us to detect copied assignments.  Please do not
  put us in a situation where we have to fail you for plagiarism.

\item[Topics:] Topics may change slightly as the course progresses.
\begin{itemize}  
\item Analysis of algorithms (informal)
\item Big-O notation
\item Sorting
  \begin{itemize}
  \item Insertion sort
\item Merge sort
\item Quick sort
\item Radix sort
  \end{itemize}
\item Graphs
  \begin{itemize}
\item Graph terminology
\item Graph data structures
\item Topological sort
\item Shortest Paths through a DAG algorithm
\item Breadth-first and depth-first traversals
\item Dijkstra's 
\item Prim's
\item PageRank
  \end{itemize}
\item Trees
  \begin{itemize}
\item Tree terminology
\item Tree data structures
\item Expression trees
\item Tree traversals (using expression trees and *-fix notation as a motivating example)
\item Coding trees and huffman trees
\item BSTs
\item AVL trees
\item Splay trees
\end{itemize}
\item Heaps
\begin{itemize}
\item Heap sort
\end{itemize}
\item Hash tables
\begin{itemize}
    \item Separate chaining
    \item Rehashing
    \item Open addressing
    \item Hash functions
\end{itemize}
\end{itemize}
\end{description}


\end{document}
